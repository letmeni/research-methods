%%%%%%%%%%% This here is for printing
%\documentclass[handout]{beamer}
%\usepackage{pgfpages}
%\pgfpagesuselayout{2 on 1}[a4paper,border shrink=5mm] % uncomment this for printing 2 pages per slide
%% General document %%%%%%%%%%%%%%%%%%%%%%%%%%%%%%%%%%

\documentclass[10pt]{beamer}   % comment for printing 

\usetheme{Boadilla} %Marbug
\useoutertheme{split} % Alternatively: miniframes, infolines, split
\useinnertheme{circles}

\definecolor{UBCblue}{rgb}{0.04706, 0.13725, 0.26667} %  Blue 
\definecolor{ao(english)}{rgb}{0.0, 0.5, 0.0} %Green 
\definecolor{darkpastelgreen}{rgb}{0.01, 0.75, 0.24}
\definecolor{darkspringgreen}{rgb}{0.09, 0.45, 0.27}


\usecolortheme[named=darkspringgreen]{structure}

\usepackage[utf8]{inputenc}
\usepackage[longnamesfirst, sort]{natbib} % For references
\usepackage[T1]{fontenc}
\usepackage{amsmath}
\usepackage{amsfonts}
\usepackage{amssymb}
\usepackage{graphicx}
%\usepackage{multicol}
\usepackage{lmodern} % load a font with all the characters
\usepackage{tcolorbox}
%\usepackage{enumitem}
\usepackage{hyperref}
\usepackage{adjustbox} 
\usepackage{booktabs} % Nice tables
\usepackage{dcolumn} % Booktabs column spacing
\usepackage[font=footnotesize,labelfont=bf]{caption}  % small fontize for figure caption
\usepackage{array}
\usepackage{tabulary}

\usepackage{color, colortbl}


\newcommand*{\captionsource}[2]{%
  \caption[{#1}]{#1}\smallskip\par
  \textbf{Source:} #2\par}

%\AtBeginSection{\frame{\sectionpage}}
\setbeamercovered{transparent}
\setbeamertemplate{caption}[numbered]
%\setbeamercovered{invisible}



\title{Research Methods for Political Science\\ PO3110 (TCD)}
\subtitle{HT: Tutorial 1 - Week 2}
% \date{\today}
\date{28-29 January 2020}
\author{Letícia Meniconi Barbabela}
%\institute{Trinity College Dublin, \\ \url{https://andrsalvi.github.io/research-methods/}}
% \titlegraphic{\hfill\includegraphics[height=1.5cm]{logo.pdf}}


% in squared brackets shows what will be on footer, usually shorter version of affiliations


\begin{document}

%%%%%%%%%%%%%%%%%%%%%%%%%%%%%%%
\begin{frame}
\titlepage
\end{frame}

%%%%%%%%%%%%%%%%%%%%%%%%%%%%%%%

\section{General Information}

\begin{frame}{Tutorials}

 \begin{itemize}

\item Participation is mandatory;
\item Come prepared: reading assigned materials, attending lectures, doing homework; 
\item Using SPSS (Download it \href{( http://isservices.tcd.ie/software/kb/student_software.php)}{here});
\item Going over homework, problem sets and topics from lectures (not a substitute!);

\end{itemize}
\end{frame}

%%%%%%%%%%%%%%%%%%%%%%%%%%%%%%%%%%

\begin{frame}{Assessment}
\begin{enumerate}
\item 60\% $\Rightarrow$ Exam;
\item 20\% $\Rightarrow$  4 homework exercises. 
\begin{itemize}
\item Submit online via Blackboard (Turnitin) on the Monday evening preceding the tutorial session;
\end{itemize}
\item 16\% $\Rightarrow$  Research project paper: 
\begin{itemize}
\item Deadline: Monday, 20/04 @ 11:59pm
\item Check assigned groups on Blackboard;
\end{itemize}
\item 4\% $\Rightarrow$  Tutorial participation, including presentation sessions (Weeks 11\&12):
\begin{itemize}
\item \textit{Two unexcused absences in tutorials will be
tolerated. Beyond that, the student will receive a zero for participation.} 
\end{itemize}
\end{enumerate}
\end{frame}

%%%%%%%%%%%%%%%%%%%%%%%%%%%%%%%%%%

\begin{frame}{Late Submission Policy}
\begin{itemize}
\item Talk to me in advance and send evidence of serious circumstances;
\item 5 points per day will be taken off your mark on assignments submitted late without a valid and previous excuse (capped at 30
points for the paper);
\item Homework exercices received after noon on Tuesdays automatically receives a zero.
\end{itemize}
\end{frame}

%%%%%%%%%%%%%%%%%%%%%%%%%%%%%%%%%%

\begin{frame}{Assignments' Submission}

    \begin{itemize}
\item Blackboard/Turnitin (not by email!);    
\item \LaTeX, Word/Open Office and submitted as \textbf{PDFs};
    \item Statistical Software: SPSS. You can use alternatives such as R or STATA if you want, but not Excel!
    \item Please do include the syntax (code) from whichever software you are using;
 \item When including tables use the “export” function from SPSS saving figures in high resolution;
 \item Don't submit screen-shots.
    \end{itemize}
\end{frame}

%%%%%%%%%%%%%%%%%%%%%%%%%%%%%%%%%%

\begin{frame}{Important Dates}


\textbf{Homework:}

\begin{itemize}
\item Week 4:   HW 1 (Monday, 10 February @ 11:59pm)
\item Week 6:   HW 2 (Monday, 24 February @ 11:59pm)
\item Week 9:   HW 3 (Monday, 23 March @ 11:59pm)
\item Week 11: HW 4 (Monday, 6 April @ 11:59pm)
\end{itemize}

\textbf{Research project:}

\begin{itemize}
\item Weeks 11 \& 12: Group presentations;
\item Monday, 20 April @ 11:59pm: Paper submission. 
\end{itemize}

\end{frame}

%%%%%%%%%%%%%%%%%%%%%%%%%%%%%%%%%%
\begin{frame}{Support}
\begin{itemize}
\item Slides from MT: \url{http://andrsalvi.github.io/research-methods}
\item Slides from our tutorials (HT): to be set up
\item Questions:
\begin{enumerate}
\item preferably in class;
\item Office hours: Tuesdays 11-12;
\item leticia.barbabela@ucdconnect.ie
\end{enumerate}
\end{itemize}

\end{frame}

%%%%%%%%%%%%%%%%%%%%%%%%%%%%%%%%%%
%%%%%%%%%%%%%%%%%%%%%%%%%%%%%%%%%%

\section{Describing a distribution}

%%%%%%%%%%%%%%%%%%%%%%%%%%%%%%%%%%



%%%%%%%%%%%%%%%%%%%%%%%%%%%%%%%%%%


\begin{frame}{Today's tutorial}


\begin{itemize}
\item Describe/summarize a \textbf{sample}: measures of central tendency and dispersion;
\item Infer something about \textbf{population};
\item \textbf{Test} a hypothesis.
\end{itemize}

\end{frame}

%%%%%%%%%%%%%%%%%%%%%%%%%%%%%%%%%%


%%%%%%%%%%%%%%%%%%%%%%%%%%%%%%%%%%

\begin{frame}{Using SPSS to describe distributions of variables in our sample}


\begin{enumerate}
    \item Open SPSS; \pause
    \item Download the dataset from James D. Fearon and David D. Laitin, "Ethnicity, Insurgency, and Civil War," American Political Science Review 97, 1 (March 2003): 75-90: 
\begin{itemize}
 \item \url{https://tinyurl.com/method-conflict}  \pause
\end{itemize}
\item Calculate mode, median, mean and standard deviation for "population":  \pause
\begin{itemize}
\item What do we conclude?  \pause
\end{itemize}
\item Do the same for "country region": \pause
\begin{itemize}
\item What is the problem?
\end{itemize}
\end{enumerate}



\end{frame}

%%%%%%%%%%%%%%%%%%%%%%%%%%%%%%%%%%


%%%%%%%%%%%%%%%%%%%%%%%%%%%%%%%%%%


\begin{frame}{Measures of Central Tendency}


 \textbf{Central Tendency:}
\begin{itemize}
\item Gives us a sense of the where to locate the "centre" of the distribution.
\end{itemize}
\textbf{Measures:}
\begin{enumerate}
\item Mode
\item Median
 \item Mean 
\end{enumerate}

\end{frame}

%%%%%%%%%%%%%%%%%%%%%%%%%%%%%%%%%%

\begin{frame}{Practical Calculations of Central Tendency}
\begin{itemize}
    \item \textbf{Mode:}
\begin{itemize}
\item The score that occurs most frequently in the data set; 
\item The tallest bar in a frequency distribution.
\end{itemize}
\item \textbf{Median:}
\begin{itemize}
 \item Rank scores according to magnitude;
  \item Choose the middle one;
  \item Odd: $\frac{n+1}{2}$
   \item Even: average between the value at position $\frac{n}{2}$ and $\frac{n+1}{2}$
\end{itemize}
\item \textbf{Mean:}
\begin{itemize}

\item Average; \\~\\ 
 \item $\bar{x}=\frac{\sum^{n}_{i=1}{x_{i}}}{n}$ \\~\\ 
 \item Influenced by outliers, while mode and median are not.

\end{itemize}
\end{itemize}

\end{frame}

%%%%%%%%%%%%%%%%%%%%%%%%%%%%%%%%%%

\begin{frame}{Measures of Dispersion}

\begin{itemize}
\item \textbf{Range}: Difference between largest and smallest observation; 
\item  \textbf{Deviance/Spread}:
\begin{itemize}
      \item \textbf{Total dispersion}: "Sum of Squared Errors (SS)"  \\~\\  $\sum(x-\bar{x})^{2}$  \\~\\ %% add space
      \item \textbf{Average dispersion}: "Variance"  \\~\\  Var(x) = $\sigma^{2}= \frac{SS}{n-1} =\frac{\sum(x-\bar{x})^{2}}{n-1}$ \\~\\  %% add space
      \item \textbf{Average dispersion squared}: "Standard deviation" \\~\\ sd(x) = $\sigma=\sqrt{\frac{\sum(x-\bar{x})^{2}}{n-1}} = \sqrt{\sigma^{2}}$  \\~\\ 
\end{itemize}
\end{itemize}
The \textbf{sample} variance is denoted by $s^2$ and the  \textbf{sample} standard deviation by $s$.

\end{frame}

%%%%%%%%%%%%%%%%%%%%%%%%%%%%%%%%%%



%\begin{frame}{Standard deviation}
%\begin{itemize}

%\item Just one of the measures of spread of data around mean (as SS, variance);
%\item What does it mean if it is large?
%\item What does it mean if it is small?
%\item What does it mean if it is 0?
%\item What does it mean in terms of representativeness of the mean?

%\end{itemize}

%\end{frame}

%%%%%%%%%%%%%%%%%%%%%%%%%%%%%%%%%%
%%%%%%%%%%%%%%%%%%%%%%%%%%%%%%%%%%

\section{Estimating a parameter from the sample}

%%%%%%%%%%%%%%%%%%%%%%%%%%%%%%%%%%
%%%%%%%%%%%%%%%%%%%%%%%%%%%%%%%%%%

\begin{frame}{The goal is inference}

More than being able to describe/summarize the \textbf{sample} (with measures of central tendency and dispersion), we want to learn something (value, relationship ...) about the \textbf{population}, for instance ...

\end{frame}

%%%%%%%%%%%%%%%%%%%%%%%%%%%%%%%%%%

\begin{frame}{Using known distributions}


\begin{itemize}
\item In statistics we have some known distributions: t-distribution, chi-square, F-distribution etc;
\begin{itemize}
\item Probability density functions;
\end{itemize}
\item \textbf{Normalizing:} Transform our data into a distribution we know, e.g.:\\~\\ 
 $z=\frac{\text{estimate}-\text{hypothesized value}}{\text{standard deviation of the estimate}}$  \\~\\ 
\item we can use this z-score to assess the probability of observing a value this extreme by chance;
\item this is more than roughly guessing the probability simply by looking at the distribution in our sample..
\end{itemize}
\end{frame}

%%%%%%%%%%%%%%%%%%%%%%%%%%%%%%%%%%

%% {Statistical models as a way to analyse a sample to learn about population}

\begin{frame}{Parameter}

\begin{itemize}
\item A numerical quantity that characterizes a given population;
\item As an example we will be looking at the mean;
\item The mean is a summary: a hypothetical value that doesn't have to be observed in the data;
    \item We use the \textbf{sample} mean ($\bar{x}$) to estimate the \textbf{population} mean ($\mu$), that is, a parameter;
    \item The sample we have is one of many possible ones (in a distribution of samples, that also can be describe by measures of central tendency and spread); 
\item Each different sample will have a different mean.
\end{itemize}


\end{frame}

%%%%%%%%%%%%%%%%%%%%%%%%%%%%%%%%%%

\begin{frame}{Central Limit Theory}
    
\begin{itemize}

\item As samples get large (> 30), the sampling distribution has a \textbf{"normal distribution"} with a mean equal to the population mean; 
\item The standard deviation of the sample means is also known as the "standard error of the mean"  or \textbf{standard error}: \\~\\  $SE_{\bar{X}}=\frac{s}{\sqrt{n}}$  


\end{itemize}

\end{frame}



%%%%%%%%%%%%%%%%%%%%%

\begin{frame}{Confidence Intervals}

\begin{itemize}

\item \textit{Limits constructed such that for a certain percentage of samples (eg.: 95\%) the true value of the population parameter will fall within these limits;} \\~\\ 
\item Confidence Interval (in this case the population parameter is the mean): \\~\\  CI = $\bar{x} \pm z *\frac{s}{\sqrt{n}}$  \\~\\
The Z score for 95\% confidence is 1.96   \\~\\
 \item It applies to other parameters as well, we are looking at the mean just as an example... \pause 
\item  Let's do it on SPSS with a different dataset...\pause
\item Calculate the confidence interval for "age"; \pause
\item What does this result tell us? \pause

\end{itemize}

\end{frame}

%%%%%%%%%%%%%%%%%%%%%%%%%%%%%%%%%%
%%%%%%%%%%%%%%%%%%%%%%%%%%%%%%%%%%

\section{Testing a hypothesis}

%%%%%%%%%%%%%%%%%%%%%%%%%%%%%%%%%%
%%%%%%%%%%%%%%%%%%%%%%%%%%%%%%%%%%

\begin{frame}{Hypothesis testing}

\begin{itemize}

\item Prediction from theory, eg.: \textit{difference with respect to a set value, relationship between variables}; 

\item Fit a model to the data and evaluate the probability of the results shown by the model given the assumption that no effect exists (null hypothesis); \\~\\

 $outcome_i=bX_i+error_i$  \\~\\

\begin{itemize}

\item Prediction: difference (one variable) or relationship (two variables)? \\~\\

\item Level of measurement of variable(s);  \\~\\

\item Choose levels of significance (eg.: 95\%);  \\~\\

\item Test statistic = $\frac{effect}{error}$  \\~\\

\item p-value: probability of getting such a test statistic score under null hypothesis.

\end{itemize}



\end{itemize}

\end{frame}

%%%%%%%%%%%%%%%%%%%%%%%%%%%%%%%%%%

%\begin{frame}{Some statistical tests}
%\begin{itemize}
%\item \textbf{One sample t-test:} Difference within one categorical variable;
%\item \textbf{$\chi^2$ and Cramer's V:} Relationship between two categorical variables;
%\item \textbf{Correlation:} Relationship between two continuous variables.
%\end{itemize}
%\end{frame}

%%%%%%%%%%%%%%%%%%%%%%%%%%%%%%%%%%

%%%%%%%%%%%%%%%%%%%%%%%%%%%%%%%%%%

\begin{frame}{One Sample T-test}

Compare the mean of a continuous variable to a specified constant value, e.g.: \textit{Do students from this class have grades higher than 75\%?} \\~\\
\begin{itemize}
\item \textit{$H_{0}$}: there isn't a difference between the observed value and the reference one;
\item \textit{$H_{1}$}: there is a difference between the observed value and the reference one;
\item Evaluate whether it is a one-tailed t-test (directional: in our example higher or lower grades) or a two-tailed one (non-directional: in our example different grades); \\~\\
 $t =\frac{\text{observed value - expected value under $H_{0}$}}{\text{standard error}}$ = $\frac{\bar{x}-m_0}{s/\sqrt{n}}$
\end{itemize}
\end{frame}

%%%%%%%%%%%%%%%%%%%%%%%%%%%%%%%%%%

\begin{frame}{$\chi^{2}$ }
Independence between two categorical variables, e.g.: \textit{Do Tuesday students wear black sweaters more often than Wednesday ones?}
\begin{itemize}
\item \textit{$H_{0}$}: $x$ is independent upon $y$ 
\item \textit{$H_{1}$}: $x$ is dependant upon $y$ 
\item We need to know 2 things: the $\chi^{2}$  score and the degrees of freedom (df): \\~\\
$\chi^{2}= \sum \frac{(f_{o}-f_{e})^2}{f_{e}}$ \\~\\
\begin{itemize}
         \item $f_{o} =$ observed frequencies  \\~\\
         \item $f_{e}= \text{expected frequencies (assuming independence)}=  \frac{\text{row margin}*\text{column margin}}{\text{total}}$  \\~\\
$df = (\text{rows}-1)*(\text{columns}-1)$ \\~\\ 
\end{itemize}
\item Tell us if we can reject the null hypothesis about independence, but nothing about the strengh of the relationship (see Cramer's V)
\end{itemize}
\end{frame}

%%%%%%%%%%%%%%%%%%%%%%%%%%%%%%%%%%

\begin{frame}{Cramer's V}
\begin{itemize}
\item Strengh of relationship between two categorical variables:  \\~\\
 $V= \sqrt{\frac{\chi^{2}}{N * \text{k-1}}}$   \\~\\
\item k = rows (r) or collumns (c), whichever is smaller;
\item If N is large, you are likely to find a significant
relationship (but it might be a weak one). 
\end{itemize}

\begin{block}{Other measures:}
Go back to MT7 slides for measures of association $\lambda$ and $\gamma$
\end{block}

\end{frame}


%%%%%%%%%%%%%%%%%%%%%%%%%%%%%%%%%%

\begin{frame}{Variance, Co-variance and Correlation}
Relationship between two numeric variables, e.g.: \textit{Is studying more hours associated to having higher grades?} : \\~\\
\begin{itemize}
    \item Var(x) = Cov(x,x) \\~\\ 
\begin{itemize}
    \item Variance: $\sigma^{2} =\frac{\sum(x-\bar{x})^{2}}{n-1} = \frac{\sum(x-\bar{x})(x-\bar{x})}{n-1}$   \\~\\ 
    \item Co-variance: $\sigma_{xy} = \frac{\sum(x-\bar{x})(y-\bar{y})}{n-1} $  \\~\\ 
    \item $\sigma_{x}\sigma_{y} \leq \sigma_{xy} \leq \sigma_{x}\sigma_{y}$   \\~\\ 
\end{itemize}
    \item Correlation: Co-variance standardized \\~\\ 
\begin{itemize}
    \item $r=\frac{\sigma_{xy}}{\sigma_{x} \times \sigma_{y}}$  \\~\\ 
    \item $-1 \leq r \leq 1$
\end{itemize}
\end{itemize}

\end{frame}

%%%%%%%%%%%%%%%%%%%%%%%

\begin{frame}{Work in pairs}

\begin{enumerate}
    \item Let's go back to analysing the conflict dataset on SPSS;
    \item Define a research question and a hypothesis;
    \item Describe the variable(s) you are interested at using plots and/or tables;
    \item Identify and perform a suitable statistical test;
    \item Present your results to your classmates.
\end{enumerate}


\end{frame}


%%%%%%%%%%%%%%%%%%%%%%%%%%%%%%%%%%

\begin{frame}{References}

\begin{itemize}
\item Field, A (2013) \textit{Discovering Statistics Using SPSS}. 4th edition. London:Sage

\item \url{http://andrsalvi.github.io/research-methods}

\end{itemize}

\end{frame}


\end{document}
